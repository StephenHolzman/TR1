

\documentclass{article}

\usepackage{amsmath}
\usepackage{natbib}
\bibpunct{(}{)}{,}{a}{}{;} 
\usepackage{nth}
\newcommand{\dd}{\; \mathrm{d}}
\usepackage{url}
\usepackage{authblk}

\begin{document}
\title{Reading Human Fertility Database and Human Mortality Database data into \texttt{R}}
\author{Tim Riffe}
\affil{Department of Demography, University of California, Berkeley}
\maketitle
\begin{abstract}
The features and usage of the \texttt{HMDHFDplus} package are demonstrated for
reading data from the Human Mortality Database, the Human Fertility Database,
and other similarly formatted sources directly from the database Websites into
\texttt{R}.
\end{abstract}

\section{Motivation}
The \citet{HFD} and \citet{HMD} are two widely used data sources
for the comparative and historical study of fertility and mortality. Both
databases offer the option to download data in bulk in a few different formats. After a bulk
download, users can then set up local databases in a variety of
convenient ways, such as that described by \citet{minton2015} for \texttt{R}
\citep{Rcitation} users. Another option for \texttt{R} users is to read data
directly into an interactive session from the respective database websites. This
is handy for small examples, lightweight reproducibility, and rapid
prototyping.
The \texttt{HMDHFDplus} package provides easy direct access to the databases using a simple standard set of arguments. Issues such as authentication and fixing column classes are handled
automatically. Analogous functions are also made
available for selected databases using similar formatting standards. At this
time, these databases include the \citet{JMD}, the \citet{CHMD}, and the
\citet{HFC}.\footnote{The \citet{HLD} may also be
incorporated in the future.} This report outlines the basic features and provides usage
examples.

\section{Installation}
The \texttt{HMDHFDplus} package is hosted on and can be installed
directly from \texttt{github}.\footnote{A package snapshot is included with
this article, but I encourage users to use the current build on
\texttt{github}, as it may contain updates and bug fixes. The package is under
the following url:
\url{http://github.com/timriffe/TR1/tree/master/TR1/HMDHFDplus}. This report,
as well as a useful \texttt{README} document, can be found under
\url{http://github.com/timriffe/TR1}.} Two external dependencies help \texttt{R}
handle \texttt{html} parsing and database authentication, and these must be installed first in order for \texttt{HMDHFDplus} to properly install.
These two dependencies are \texttt{cURL} and \texttt{XML}, and their
installation unfortunately depends on one's operating system at this time.

In \texttt{Linux} and similar systems, open the Terminal and run:
\begin{verbatim}
sudo apt-get install libcurl
sudo apt-get install libxml2-dev
\end{verbatim}
When these external dependencies are properly installed, you should install
\texttt{R}'s \texttt{XML} and \texttt{RCurl} packages in the usual way. Then install \texttt{HMDHFDplus} using:
\begin{verbatim}
library(devtools)
install_github("timriffe/TR1/TR1/HMDHFDplus")
\end{verbatim}
For \texttt{Windows} and \texttt{Mac}, installation may be simpler
yet.\footnote{Not all versions of Windows and Mac have been tested.} Simply be sure to install the
most recent version of \texttt{R}, then install the \texttt{devtools} package,
then run the above two lines of \texttt{R} code.

Please consult the \texttt{README}\footnote{The \texttt{README} is at the
foot of the main repository
page: \url{https://github.com/timriffe/TR1}} file on the
\texttt{github} repository for more details. 

\section{Usage and examples}
Load the package using:
\begin{verbatim}
library(HMDHFDplus)
\end{verbatim}
The two main functions of interest are \texttt{readHMDweb()} and
\texttt{readHFDweb()}, and both functions have the same essential
arguments. These two functions only require the user to supply country codes,
data item names, and database authentication parameters.\footnote{The HFD also
allows users to extract data from former updates, which may be useful for
strict replication purposes. In order to make use of this feature, users must
note the 8-digit date code associated with the specific country series
update. By default, \texttt{readHFDweb()} extracts the most recent update.} It
helps to be familiar with HMD and HFD file naming conventions. To retrieve the population codes used in any of
these databases, run:
\begin{verbatim}
getHFDcountries()
getHFCcountries()
getHMDcountries()
getJMDprefectures()
getCHMDprovinces()
\end{verbatim}
This returns vectors of the standard numerical or letter codes used to identify
population units. 

The functions used for reading data into \texttt{R} from the Web use a
common set of required arguments. In interactive \texttt{R} sessions, the
following will prompt the user to enter a username and password into the
console (no quotes) each time the function is run:
\begin{small}
\begin{verbatim}
# for HMD:
USmales <- readHMDweb(CNTRY = "USA", item = "mltper_1x1")
# for HFD (will need to re-enter username and password)
USfert <- readHFDweb(CNTRY = "USA", item = "asfrRR")
\end{verbatim}
\end{small}
Manually entering a username and password can become tedious for larger
data-grabs, so these can also be given explicitly in the arguments, like so:
\begin{small}
\begin{verbatim}
USmales <- readHMDweb(CNTRY = "USA", item = "mltper_1x1", username =
"myusername", password = "mypassword")
USfert <- readHFDweb(CNTRY = "USA", item = "asfrRR", username =
"myusername", password = "mypassword")
\end{verbatim}
\end{small}
There is a security trade-off in this case, because the username and password
may inadvertently be saved within your \texttt{R} script. I suggest two
alternatives in this case. First, in an interactive \texttt{R} session, define your username once at
the beginning of the script, but without saving them as text within the script,
like so:
\begin{small}
\begin{verbatim}
pw <- userInput()
us <- userinput()
USmales <- readHMDweb(CNTRY = "USA", item = "mltper_1x1", username = us,
password = pw)
\end{verbatim}
\end{small}
The two objects \texttt{pw} and \texttt{us} can in this case be recycled
throughout the following \texttt{R} session. Second, For more frequent users, I
recommend defining your HMD and HFD passwords in the \texttt{.Rprofile} file,
such that they are defined and ready to use at the start of \texttt{R}
sessions, but are not saved in your potentially-shared code. The above HMD
code will return data such as the following:
\texttt{data.frame}:
\begin{small}
\begin{verbatim}
head(USmales)
  Year Age      mx      qx   ax     lx   dx    Lx      Tx    ex OpenInterval
1 1933   0 0.06859 0.06515 0.23 100000 6515 94978 5916978 59.17        FALSE
2 1933   1 0.01004 0.00999 0.50  93485  934 93018 5822000 62.28        FALSE
3 1933   2 0.00467 0.00466 0.50  92551  431 92336 5728982 61.90        FALSE
4 1933   3 0.00333 0.00333 0.50  92120  307 91967 5636646 61.19        FALSE
5 1933   4 0.00254 0.00253 0.50  91814  233 91697 5544679 60.39        FALSE
6 1933   5 0.00209 0.00209 0.50  91581  191 91485 5452982 59.54        FALSE
\end{verbatim}
\end{small}
This \texttt{data.frame} differs from the original HMD \texttt{mltper\_1x1} file
in that the \texttt{Age} column is integer, and a new \texttt{OpenInterval}
column has been added, which contains the value \texttt{TRUE} for age 110. HFD
\texttt{Age} and \texttt{Cohort} columns are modified in a similar way for more
intuitive and immediate use of these columns as integers. Likewise, abridged
ages, such as \texttt{"5-9"} are coerced as integers of the
lower interval bound, as \texttt{5}. Finally, HMD \texttt{Population} files, obtained via
\begin{small}
\begin{verbatim}
USpop <- readHMDweb("USA","Population",username = us, password = pw)
head(USpop)
  Year Age OpenInterval   Female1   Male1  Total1   Female2     Male2  Total2
1 1933   0        FALSE  984472.3 1015362 1999834  937185.8  968955.4 1906141
2 1933   1        FALSE 1040496.0 1064088 2104584  970696.5  993352.8 1964049
3 1933   2        FALSE 1093043.8 1117527 2210571 1062002.5 1083452.4 2145455
4 1933   3        FALSE 1107994.3 1135047 2243041 1095555.1 1121220.2 2216775
5 1933   4        FALSE 1130624.4 1179514 2310138 1105999.3 1132665.9 2238665
6 1933   5        FALSE 1168930.6 1228225 2397156 1141944.4 1197735.4 2339680
\end{verbatim}
\end{small}
, where columns ending in 1 indicate January \nth{1} estimates and columns
ending in 2 indicate December \nth{31} estimates, and the \texttt{Year} and
\texttt{Age} columns are coerced to an integer class. The JMD, CHMD, and HFC
are all called in similar ways, but without authentication:
\begin{small}
\begin{verbatim}
# 31 columns!
USasfrBO <- readHFC("USA","ASFRstand_BO")
# 5x5 male lifetables for Aomori prefecture:
Aomori <- readJMDweb("02","mltper_5x5")
# 5x5 lifetables for Alberta:
ALB <- readCHMDweb("alb","mltper_5x5")
\end{verbatim}
\end{small}
The JMD and CHMD follow the same formatting standards and naming conventions as
the HMD, although the data products available are a subset of those produced by
the HMD. The HFC follows different standards and conventions than the HFD.

\section{Conclusions}
Reading data directly from HMD, HFD and a selection of other databases directly
from the web into \texttt{R} is made easy with the \texttt{HMDHFDplus} package. At this time, utilities are provided for reading data from the HMD,
JMD, CHMD, HFD, and HFC websites. Common
\texttt{R} pitfalls are removed by coercing columns to useful classes by
default.

\section{Acknowledgements}
Thanks to Joshua Goldstein and Carl Boe for supporting development of this R
functionality, and to Vladimir Shkolnikov, Dmitri Jdanov, and Tom\'{a}\^{s}
Sobotka for the invitation to present this material at the HFD side meeting to
the 2015 PAA Annual Meeting.
This work was supported by the National Institute On Aging of the U.S. National Institutes of Health (NIH) under Award Numbers R01-AG011552 and R01-AG040245. The content is solely the responsibility of the
author and does not necessarily represent the official views of the NIH.

\bibliographystyle{plainnat}
  \bibliography{references} 

\end{document}

